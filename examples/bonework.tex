%!TEX program = xelatex
\documentclass[cn,hazy,blue,14pt,normal]{elegantnote}

\usepackage{array}
\usepackage{amsmath}
\usepackage{tikz}
\usepackage{tkz-fct}
\usepackage{tkz-euclide}% 绘制线段命令(此处仅标注尺寸)
\usepackage{subfig}
\usepackage{overpic}

\usetikzlibrary{shadows}
\usetikzlibrary{intersections,decorations.text}
\usetikzlibrary{dsp,chains}
\DeclareMathAlphabet{\mathpzc}{OT1}{pzc}{m}{it}
\newcommand{\z}{\mathpzc{z}}
\newcommand{\sgn}{\text{sgn}}
\usetikzlibrary{
    shapes.geometric,   %几何形状
    calc,               %计算
    positioning         %用于定位
}
\usetikzlibrary{arrows,arrows.meta}
\tikzset{
    every picture/.style={
        >=latex,
        node distance=5mm and 5mm
        %有无and的区别见手册positioning部分
}}
\input{main.tikzstyles}
\newtheorem{train}{实验}

\begin{document}
\includepdf{cover.pdf}

\section{实验配置}
\begin{itemize}
    \item 程序测试系统:Windows 10
    \item 程序运行环境:Python3.7
    \item Python库(标准库未列出):numpy, sklearn
    \item 报告编写环境:Bone\TeX{}, \TeX{} Live 2020
    \item 开发工具:PyCharm, VSCode
\end{itemize}

\section{数据集介绍}
Iris 数据集是模式识别中最著名的数据集之一。Iris 数据集包含 3 个
类,每个类有 50 个实例,其中每一类都是指一种鸢尾属植物。有一类是
与另外两类是线性可分的,而另外两类之间是线性不可分的。

Iris 数据集是非常经典的多分类数据集,其参数如表\ref{tab:3078999826376}

\begin{table}[htbp]
    \centering
    \caption{iris数据集参数}
    \label{tab:3078999826376}
    \begin{tabular}{|c|c|}
        \hline
        类别数     & 3   \\
        \hline
        每类样本数 & 50  \\
        \hline
        样本总数   & 150 \\
        \hline
        维度       & 4   \\
        \hline
    \end{tabular}
\end{table}

\section{算法原理与实现}
\subsection{贝叶斯分类}
\subsubsection{算法原理}
贝叶斯分类的核心原理是贝叶斯定理,即
\begin{equation}
    P(A|B) = \frac{P(B|A)P(A)}{P(B)}
\end{equation}

首先假设有$N$种可能的分类类别,
分别为$\{c_1,c_2, \cdot ,c_N\}$,这时候对于样本$x$,
利用贝叶斯
\begin{equation}
    P(c|x) = \frac{P(x|c)P(c)}{P(x)}
\end{equation}
找到最大的$P(c|x)$作为样本$x$的最终分类,即
\begin{equation}
    c^* = \arg\max_c\frac{P(x|c)P(c)}{P(x)}
\end{equation}
$P(c)$很容易就可以得到,
令$D_c$表示训练集$D$中第$c$类样本组成的集合,
若有充足的独立同分布样本,
则可以很容易估计出$P(c)$,即
\begin{equation}
    P(c) = \frac{ |D_c| }{D}
\end{equation}
不难发现,基于贝叶斯公式,
主要是$P(x|c)$比较难以求解,
一般估计类条件概率主要使用极大似然估计法,
但因为这里是所有属性上的联合概率,
难以运用最大似然估计法直接估计得到。
为了避开这个障碍,
朴素贝叶斯分类器运用了“属性条件独立性假设”:
对已知类别,假设所有属性相互独立。
\begin{equation}
    P(c) = \frac{ |D_c| }{|D|}
\end{equation}

基于上述假设,贝叶斯分类公式可以重写为:
\begin{equation}
    P(c|x) = \frac{P(c)}{P(x)}\prod_{i = 1}^d P(x_i|c)
\end{equation}
其中$d$为属性的数目,$x_i$为样本$x$在第$i$个属性上的取值。

由于对所有的类别来讲,$P(x)$都相同,
所以基于上式的贝叶斯判定准则
\begin{equation}
    c^* = \arg\max_c P(c)\prod_{i = 1}^d P(x_i|c)
\end{equation}

对于离散属性而言,
令$D_{cx_i}$表示在第$i$个属性上取值为$x_i$的样本组成的集合,
则类条件概率可以估计
\begin{equation}
    P(c) = \frac{ |D_{cx_i}| }{D}
\end{equation}


\subsection{高斯判别分析}
\subsubsection{算法原理}
高斯判别分析属于机器学习算法中的分类算法,
不妨假设样本数据为两种类别,
它的大致思想是通过两个先验假设:
一是样本数据的类别$y$在给定的情况下服从伯努利分布,
二是不同类别中的样本数据分别服从多元高斯分布。
首先估计出先验概率以及多元高斯分布的均值和协方差矩阵,
然后再由贝叶斯公式求出一个新样本分别属于两类别的概率,
预测结果取概率值大者。本文推导高斯判别分析算法的流程是,
首先简单的推导出多元高斯分布、
其次提出高斯判别分析算法的假设函数、然后构造损失函数、
最后求解损失函数得出假设函数中的参数值。

正态分布$X\sim N(\mu, \sigma^2)$ ,
它的概率密度函数$\phi(x)$ 为
\begin{equation}
    \phi(x) = \frac{1}{\sigma \sqrt{2\pi}}
    e^{ -\frac{(x -\mu)^2}{2\sigma^2} }
\end{equation}

多元正态分布$X\sim N(\mu, \Sigma)$ ,
它的概率密度函数$\phi(x)$ 为
\begin{equation}
    \phi(x) = \frac{1}{ \sqrt{2\pi} |\Sigma|^{\frac{1}{2}}}
    e^{ -\frac{1}{2}(x -\mu)^T{\Sigma^{ - 1}}(x -\mu) }
\end{equation}

不妨假设含有$m$个样本数据,
写成分布函数的形式即
\begin{equation}
    p(y) = \phi^y +( 1-\phi)^(1 -y)
\end{equation}
\begin{equation}
    p(x|y = 0) = \frac{1}{ \sqrt{2\pi} |\Sigma|^{\frac{1}{2}}}
    e^{ -\frac{1}{2}(x -\mu_0)^T{\Sigma^{ - 1}}(x -\mu_0) }
\end{equation}
\begin{equation}
    p(x|y = 1) = \frac{1}{ \sqrt{2\pi} |\Sigma|^{\frac{1}{2}}}
    e^{ -\frac{1}{2}(x -\mu_1)^T{\Sigma^{ - 1}}(x -\mu_1) }
\end{equation}
上述模型中的未知参数为$\phi, \Sigma, \mu_0, \mu_1 $。

已知样本数据含有参数的概率分布,
根据统计学的最大似然估计可以推导高斯判别分析模型的损失函数

\begin{equation}
    L(\phi, \mu_0, \mu_1, \Sigma) =
    \sum_{i = 1}^m y^{(i)}\ln p(x^{(i)}|y^{(i)} = 1)
    +\sum_{i = 1}^m (1 -y^{(i)})\ln p(x^{(i)}|y^{(i)} = 0)
    +\sum_{i = 1}^m \ln p(y^{(i)})
\end{equation}
对最大似然函数 $L(\phi, \mu_0, \mu_1, \Sigma)$ 求偏导,
令其相应偏导数为零即可求出参数
\begin{equation}
    \begin{cases}
        \phi = \frac{\sum_{i = 1}^m I(y^{(i)} = 1)}{m} \\
        \mu_0 = \frac{\sum_{i = 1}^m I(y^{(i)} = 0)x^{(i)}}
        {\sum_{i = 1}^m I(y^{(i)} = 0)}                \\
        \mu_1= \frac{\sum_{i = 1}^m I(y^{(i)} = 1)x^{(i)}}
        {\sum_{i = 1}^m I(y^{(i)} = 1)}                \\
        \Sigma = \frac{1}{m}\sum_{i = 1}^m (x -\mu_{y^{(i)}})(x -\mu_{y^{(i)}})^T
    \end{cases}
\end{equation}

\subsubsection{代码实现}
高斯判别分析代码如下
\lstinputlisting[
    style       =   python,
    label       =   {逻辑回归}
]{code/algorithm/gda.py}
\subsection{逻辑回归}
\subsubsection{算法原理}
逻辑回归的过程可以概括为:面对一个回归或者分类问题,
建立代价函数,然后通过优化方法迭代求解出最优的模型参数,
然后测试验证我们这个求解的模型的好坏。
Logistic Regression 通过 Sigmoid 函数预测样本属于正类的概率,
其预测模型为
\begin{equation}
    \hat{y} = \frac{1}{1 + e^{ - (\omega^Tx + b)}}
\end{equation}

损失函数为
\begin{equation}
    L(\omega, b) = y \ln{\hat{y}} +(1 - y)\ln{(1 -\hat{y})}
\end{equation}

通过优化损失函数使得损失最小化, 即
\begin{equation}
    \min_{\omega, b} L(\omega, b)
\end{equation}

可以采用梯度下降法进行优化,即
\begin{equation}
    \begin{cases}
        \omega = \omega - \frac{\partial L}{\partial \omega} \\
        b = b- \frac{\partial L}{\partial b}
    \end{cases}
\end{equation}
此外,优化算法还包括
\begin{itemize}
    \item   Newton Method(牛顿法)
    \item Conjugate gradient method(共轭梯度法)
    \item Quasi-Newton Method(拟牛顿法)
    \item BFGS Method
    \item L-BFGS(Limited-memory BFGS)
    \item 其他
\end{itemize}




\subsubsection{代码实现}
为了使代码编写更规范,我们首先定义一个 BoneModel 类,
本文实现的所有模型均基于此类,代码如下
\lstinputlisting[
    style       =   python,
    label       =   {模型}
]{code/algorithm/model.py}

逻辑回归代码如下
\lstinputlisting[
    style       =   python,
    label       =   {逻辑回归}
]{code/algorithm/lr.py}



\subsection{支持向量机}
\subsubsection{算法原理}
特征空间中任意一点 $x$ 到超平面的距离
\begin{equation}
    r = \frac{g(x)}{ \Vert \boldsymbol{w} \Vert }
\end{equation}
离分类面最近的样本满足$\Vert g(x) = 1\Vert$,
所以分类间隔为 $2\Vert \boldsymbol{w}\Vert$。

线性 SVM 的优化问题可以表示为
\begin{equation}
    \label{lsvm}
    \begin{aligned}
         & \min_{w,b} \frac{1}{2} \Vert \boldsymbol{w} \Vert^2\,                   \\
         & \text{s.t.}\, y_i(\boldsymbol{w} \cdot\boldsymbol{x_i} + b) - 1 \geq  0
    \end{aligned}
\end{equation}
过两类样本中离分类面最近的点,
且平行于最优超平面的超平面 $H_1$ 和 $H_2$ 上的训练样本,
就是使式\ref{lsvm}等号成立的样本,也叫支持样本。
线性情况下最优分类面的决策函数
\begin{equation}
    \begin{cases}
        f(\boldsymbol{x})   = \sgn[g (\boldsymbol{x})]
        = \sgn[\boldsymbol{w^\ast} \cdot \boldsymbol{x} + b^\ast] \\
        \boldsymbol{w^\ast} =\sum^N_{i = 1}\alpha^\ast_i y_i      \\
        b^\ast = y_j - \sum^N_{i = 1}\alpha^\ast_i y_i \boldsymbol{x_i}\cdot\boldsymbol{x_j}
    \end{cases}
\end{equation}

非线性SVM需要添加一个松弛因子 $\xi_i $,
使得 $y_i[\boldsymbol{w} \cdot\boldsymbol{x_i} + b] - 1 + \xi_i  \geq 0$成立。
线性不可分下的优化目标
\begin{equation}
    \label{usvm}
    \min_{ \boldsymbol{w},b} \frac{1}{2} \Vert  \boldsymbol{w} \Vert^2 +
    C\sum^N_{i = 1}\xi_i
\end{equation}
非线性支撑矢量机的决策函数
\begin{equation}
    f(\boldsymbol{x}) = \sgn[\sum^N_{i = 1}\alpha_i y_i
    K(\boldsymbol{x_i}, \boldsymbol{x}) + b]
\end{equation}
其中$K(\boldsymbol{x_i}, \boldsymbol{x_j})=\boldsymbol{x_i}\cdot \boldsymbol{x_j}$。

\subsubsection{算法伪代码}
SMO伪代码如算法\ref{fit}
\begin{algorithm}
    \caption{SMO算法}
    \label{fit}
    \begin{algorithmic}[1] %每行显示行号  
        \Require 数据集$\{(x^{(i)},y^{(i)}), i=1\dots m\}$,调和系数$C$,容差$t$,最大迭代数$M$
        \Ensure 乘子$\alpha$,阈值$b$
        \Function {KMeansFit}{$D, C, t, M$}
        \State $\alpha_i=0, i=1\dots m,b=0$
        \State $p=0, \epsilon=10^{-5}$

        \While {$p<M$}
        \State $c_\alpha=0$
        \For {$i=1:m$}
        \State $E_i= \sum^m_{k=1} \alpha_k
            y^{(k)}\langle x^{(k)},x^{(i)}\rangle -y^{(i)}$
        \If {$(y^{(i)}E_i<-t \&\& \alpha_i<C)||
                (y^{(i)}E_i>t \&\& \alpha_i>0)$}
        \State 随机选择 $j \neq i$
        \State $E_i= \sum^m_{k=1} \alpha_k
            y^{(k)}\langle x^{(k)},x^{(j)}\rangle -y^{(j)}$
        \State $\alpha_i^{(o)}=\alpha_i$
        \State $\alpha_j^{(o)}=\alpha_j$

        \State $L=
            \begin{cases}
                \max(0,\alpha_j-\alpha_i), y^{(i)}\neq y^{(j)} \\
                \max(0,\alpha_j+\alpha_i-C), y^{(i)}= y^{(j)}
            \end{cases}
        $
        \State $H=
            \begin{cases}
                \min(C,C+\alpha_j-\alpha_i), y^{(i)}\neq y^{(j)} \\
                \min(C,\alpha_j+\alpha_i), y^{(i)}= y^{(j)}
            \end{cases}
        $
        \If {$L=H$}
        \State continue
        \EndIf
        \State $
            \eta =2\langle x^{(i)},x^{(j)}\rangle
            -\langle x^{(i)},x^{(i)}\rangle
            -\langle x^{(j)},x^{(j)}\rangle
        $
        \If {$\eta\geq 0$}
        \State continue
        \EndIf

        \State $
            \alpha_j=\alpha_j-\frac{y^{(j)}(E_i-E_j)}{\eta}
        $
        \State $
            \alpha_j=
            \begin{cases}
                H, \alpha_j>H                    \\
                \alpha_j,  L\leq \alpha_j \leq H \\
                L, \alpha_j <L
            \end{cases}
        $
        \algstore{myalg}
    \end{algorithmic}
\end{algorithm}
\begin{algorithm}
    \begin{algorithmic}[1]
        \algrestore{myalg}
        \If {$|\alpha_j-\alpha_j^o|<\epsilon$}
        \State continue
        \EndIf
        \State $\alpha_i=\alpha_i^o+
            y^{(j)}y^{(j)} (\alpha_i^o-\alpha_i)$
        \State $b_1=b^o-E_i+
            y^{(i)}(\alpha_i^o-\alpha_i)\langle x^{(i)},x^{(i)}\rangle
            +y^{(j)}(\alpha_j^o-\alpha_j)\langle x^{(i)},x^{(j)}\rangle$
        \State $b_2=b^o-E_i+
            y^{(i)}(\alpha_i^o-\alpha_i)\langle x^{(i)},x^{(j)}\rangle
            +y^{(j)}(\alpha_j^o-\alpha_j)\langle x^{(j)},x^{(j)}\rangle$
        \State $b
            \begin{cases}
                b_1, 0<\alpha_i<C \\
                b_2, 0<\alpha_j<C \\
                \frac {b_1+b_2}{2}, \text{otherwise}
            \end{cases}
        $
        \State $c_\alpha=c_\alpha+1$
        \EndIf
        \EndFor

        \If {$c_\alpha=0$}
        \State $p=p+1$
        \Else
        \State $p=0$
        \EndIf
        \EndWhile
        \EndFunction
    \end{algorithmic}
\end{algorithm}
\newpage

\section{结果分析}
本文将数据集分割成训练集和测试集,
并在训练时进行五折交叉验证,
每一种算法进行了十次实验。
{{!asd}}
\subsection{性能分析}
\subsubsection{sklearn 逻辑回归}
\begin{train}交叉验证平均准确率:90.00\% \\  训练集准确率:88.89\% \\  测试集准确率:93.33\% \end{train}
\begin{train}交叉验证平均准确率:90.00\% \\  训练集准确率:88.89\% \\  测试集准确率:93.33\% \end{train}
\begin{train}交叉验证平均准确率:90.00\% \\  训练集准确率:88.89\% \\  测试集准确率:93.33\% \end{train}
\begin{train}交叉验证平均准确率:90.00\% \\  训练集准确率:88.89\% \\  测试集准确率:93.33\% \end{train}
\begin{train}交叉验证平均准确率:90.00\% \\  训练集准确率:88.89\% \\  测试集准确率:93.33\% \end{train}
\begin{train}交叉验证平均准确率:90.00\% \\  训练集准确率:88.89\% \\  测试集准确率:93.33\% \end{train}
\begin{train}交叉验证平均准确率:90.00\% \\  训练集准确率:88.89\% \\  测试集准确率:93.33\% \end{train}
\begin{train}交叉验证平均准确率:90.00\% \\  训练集准确率:88.89\% \\  测试集准确率:93.33\% \end{train}
\begin{train}交叉验证平均准确率:90.00\% \\  训练集准确率:88.89\% \\  测试集准确率:93.33\% \end{train}
\begin{train}交叉验证平均准确率:90.00\% \\  训练集准确率:88.89\% \\  测试集准确率:93.33\% \end{train}
\subsubsection{sklearn 高斯朴素贝叶斯}
\begin{train}交叉验证平均准确率:87.78\% \\  训练集准确率:86.67\% \\  测试集准确率:93.33\% \end{train}
\begin{train}交叉验证平均准确率:87.78\% \\  训练集准确率:86.67\% \\  测试集准确率:93.33\% \end{train}
\begin{train}交叉验证平均准确率:87.78\% \\  训练集准确率:86.67\% \\  测试集准确率:93.33\% \end{train}
\begin{train}交叉验证平均准确率:87.78\% \\  训练集准确率:86.67\% \\  测试集准确率:93.33\% \end{train}
\begin{train}交叉验证平均准确率:87.78\% \\  训练集准确率:86.67\% \\  测试集准确率:93.33\% \end{train}
\begin{train}交叉验证平均准确率:87.78\% \\  训练集准确率:86.67\% \\  测试集准确率:93.33\% \end{train}
\begin{train}交叉验证平均准确率:87.78\% \\  训练集准确率:86.67\% \\  测试集准确率:93.33\% \end{train}
\begin{train}交叉验证平均准确率:87.78\% \\  训练集准确率:86.67\% \\  测试集准确率:93.33\% \end{train}
\begin{train}交叉验证平均准确率:87.78\% \\  训练集准确率:86.67\% \\  测试集准确率:93.33\% \end{train}
\begin{train}交叉验证平均准确率:87.78\% \\  训练集准确率:86.67\% \\  测试集准确率:93.33\% \end{train}
\subsubsection{个人实现逻辑回归}
\begin{train}交叉验证平均准确率:81.11\% \\  训练集准确率:81.11\% \\  测试集准确率:88.33\% \end{train}
\begin{train}交叉验证平均准确率:62.22\% \\  训练集准确率:88.89\% \\  测试集准确率:91.67\% \end{train}
\begin{train}交叉验证平均准确率:62.22\% \\  训练集准确率:85.56\% \\  测试集准确率:96.67\% \end{train}
\begin{train}交叉验证平均准确率:72.22\% \\  训练集准确率:25.56\% \\  测试集准确率:30.00\% \end{train}
\begin{train}交叉验证平均准确率:77.78\% \\  训练集准确率:82.22\% \\  测试集准确率:88.33\% \end{train}
\begin{train}交叉验证平均准确率:78.89\% \\  训练集准确率:84.44\% \\  测试集准确率:88.33\% \end{train}
\begin{train}交叉验证平均准确率:76.67\% \\  训练集准确率:82.22\% \\  测试集准确率:85.00\% \end{train}
\begin{train}交叉验证平均准确率:80.00\% \\  训练集准确率:77.78\% \\  测试集准确率:71.67\% \end{train}
\begin{train}交叉验证平均准确率:86.67\% \\  训练集准确率:85.56\% \\  测试集准确率:91.67\% \end{train}
\begin{train}交叉验证平均准确率:71.11\% \\  训练集准确率:56.67\% \\  测试集准确率:56.67\% \end{train}
\subsubsection{个人实现高斯判别分析}
\begin{train}交叉验证平均准确率:57.78\% \\  训练集准确率:56.60\% \\  测试集准确率:58.17\% \end{train}
\begin{train}交叉验证平均准确率:57.41\% \\  训练集准确率:56.60\% \\  测试集准确率:58.17\% \end{train}
\begin{train}交叉验证平均准确率:58.21\% \\  训练集准确率:56.60\% \\  测试集准确率:58.17\% \end{train}
\begin{train}交叉验证平均准确率:58.09\% \\  训练集准确率:56.60\% \\  测试集准确率:58.17\% \end{train}
\begin{train}交叉验证平均准确率:57.65\% \\  训练集准确率:56.60\% \\  测试集准确率:58.17\% \end{train}
\begin{train}交叉验证平均准确率:56.42\% \\  训练集准确率:56.60\% \\  测试集准确率:58.17\% \end{train}
\begin{train}交叉验证平均准确率:56.42\% \\  训练集准确率:56.60\% \\  测试集准确率:58.17\% \end{train}
\begin{train}交叉验证平均准确率:56.17\% \\  训练集准确率:56.60\% \\  测试集准确率:58.17\% \end{train}
\begin{train}交叉验证平均准确率:56.05\% \\  训练集准确率:56.60\% \\  测试集准确率:58.17\% \end{train}
\begin{train}交叉验证平均准确率:58.02\% \\  训练集准确率:56.60\% \\  测试集准确率:58.17\% \end{train}


\section{心得体会}
机器学习是一门多领域交叉学科,
学习机器学习可以巩固概率论、统计学、算法理论等
许多门学科的基础知识,
同时还能提前了解逼近论、凸分析等多门学科的内容,
是人工智能专业本科生必须掌握的重要学科。

\appendix
\section{其他代码}
\subsection{数据集}
\lstinputlisting[
    style       =   python,
    label       =   {交叉验证}
]{code/dataset.py}
\subsection{训练和测试}
\lstinputlisting[
    style       =   python,
    label       =   {交叉验证}
]{code/main.py}

\subsection{交叉验证}
\lstinputlisting[
    style       =   python,
    label       =   {交叉验证}
]{code/utils/kflod.py}

\subsection{one-hot 编码}
\lstinputlisting[
    style       =   python,
    label       =   {oh}
]{code/utils/oh.py}

\end{document}
